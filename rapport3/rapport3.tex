\documentclass[a4paper,10pt]{report}
\usepackage[utf8]{inputenc}
\usepackage{graphicx}
\usepackage{titling}
\usepackage{fullpage}

% Title Page
\title{Rendu 4 Projet 2}
\author{Louis Bethune et Jo\"el Felderhoff}
\date{}

\begin{document}
\maketitle

\begin{section}{Notre interpreteur/compilateur fouine}

\subsection{Travail effectué tout au long du semestre}
Tout au long de ce semestre, nous avons fait divers rapports pour expliciter le travail effectué.

Pour plus de précision, se référer à \cite{Rapport1} \cite{Rapport2}

Voici la liste (non exhaustive) des choses gérées par notre interpretteur/compilateur fouine à ce jour.

\begin{itemize}
\item \textbf{Éléments de bases de caml} : fonction récursives, à plusieurs arguments ou non... arithmétique classique
\item \textbf{Aspects impératifs} : références sur tous les types, tableaux, paires, commandes lancées les unes après les autres...
\item \textbf{Exceptions} : gestion des exceptions par continuation dans le code de l'interpreteur
\item \textbf{Compilation vers machine à pile} : se référer à \cite{Rapport1}
\item \textbf{Transformation de programmes} : tranformation fouine + impératif + exception -> fouine, mais sans la gestion des fonctions récursives. Se reporter à \cite{Rapport2}
\item \textbf{Spécification de types} : on peut spécifier un type en particulier à la création des variables et le programme sera type checké avant exécution. Les variables non typées à leur création seront supposées appartenir au bon type...
\end{itemize}

\subsection{Ce que nous avons appris ce semestre}
Il semble clair que nous avons gagné une meilleure compréhension du langage OCaml ainsi que du lexer/parser associé.

Egalement, nos connaissances en conception logicielles ont été mises à rude épreuve, pour faire évoluer un projet sur un semestre entier, en rajoutant des couches sans toucher à ce qui a été fait les semaines precedentes.

Notre facultée de résistance au sommeil a été evidemment un point crucial.

Nous avons aussi beaucoup appris à travailler en équipe (facultée nommée par Louis sous le nom ``Diplomatie''). La répartition du travail n'a pas été facile à chaques fois, Jo\"el ne suivant pas les cours de Projet2 en eux meme. Néanmoins nous avons réussi à outrepasser ces difficultées, au pris d'un certain nombre d'engueulades ("Ma solution était la meilleure ! on a fait la tienne parce qu'elle était moins casse-couille !'').

Au final, on s'est bien amusé, c'était cool.

Nous ne retiendrons que deux citation : 

\begin{quote}
La prochaine fois que je vois une putain de fouine, je lui casse la gueule
\end{quote}

\begin{quote}
Bon, si a l'issue de 3 tentatives je n'arrive toujours pas à faire compiler, je prend mon ordi et je le fracasse contre un coin de table
\end{quote}

Nous laissons à la postérité deviner lequel de nous deux à dit quelle phrase.

\end{section}

\bibliographystyle{plain}
\bibliography{biblio}

\end{document}
